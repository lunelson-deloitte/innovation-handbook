% Options for packages loaded elsewhere
\PassOptionsToPackage{unicode}{hyperref}
\PassOptionsToPackage{hyphens}{url}
%
\documentclass[
]{book}
\usepackage{amsmath,amssymb}
\usepackage{lmodern}
\usepackage{iftex}
\ifPDFTeX
  \usepackage[T1]{fontenc}
  \usepackage[utf8]{inputenc}
  \usepackage{textcomp} % provide euro and other symbols
\else % if luatex or xetex
  \usepackage{unicode-math}
  \defaultfontfeatures{Scale=MatchLowercase}
  \defaultfontfeatures[\rmfamily]{Ligatures=TeX,Scale=1}
\fi
% Use upquote if available, for straight quotes in verbatim environments
\IfFileExists{upquote.sty}{\usepackage{upquote}}{}
\IfFileExists{microtype.sty}{% use microtype if available
  \usepackage[]{microtype}
  \UseMicrotypeSet[protrusion]{basicmath} % disable protrusion for tt fonts
}{}
\makeatletter
\@ifundefined{KOMAClassName}{% if non-KOMA class
  \IfFileExists{parskip.sty}{%
    \usepackage{parskip}
  }{% else
    \setlength{\parindent}{0pt}
    \setlength{\parskip}{6pt plus 2pt minus 1pt}}
}{% if KOMA class
  \KOMAoptions{parskip=half}}
\makeatother
\usepackage{xcolor}
\usepackage{color}
\usepackage{fancyvrb}
\newcommand{\VerbBar}{|}
\newcommand{\VERB}{\Verb[commandchars=\\\{\}]}
\DefineVerbatimEnvironment{Highlighting}{Verbatim}{commandchars=\\\{\}}
% Add ',fontsize=\small' for more characters per line
\usepackage{framed}
\definecolor{shadecolor}{RGB}{248,248,248}
\newenvironment{Shaded}{\begin{snugshade}}{\end{snugshade}}
\newcommand{\AlertTok}[1]{\textcolor[rgb]{0.94,0.16,0.16}{#1}}
\newcommand{\AnnotationTok}[1]{\textcolor[rgb]{0.56,0.35,0.01}{\textbf{\textit{#1}}}}
\newcommand{\AttributeTok}[1]{\textcolor[rgb]{0.77,0.63,0.00}{#1}}
\newcommand{\BaseNTok}[1]{\textcolor[rgb]{0.00,0.00,0.81}{#1}}
\newcommand{\BuiltInTok}[1]{#1}
\newcommand{\CharTok}[1]{\textcolor[rgb]{0.31,0.60,0.02}{#1}}
\newcommand{\CommentTok}[1]{\textcolor[rgb]{0.56,0.35,0.01}{\textit{#1}}}
\newcommand{\CommentVarTok}[1]{\textcolor[rgb]{0.56,0.35,0.01}{\textbf{\textit{#1}}}}
\newcommand{\ConstantTok}[1]{\textcolor[rgb]{0.00,0.00,0.00}{#1}}
\newcommand{\ControlFlowTok}[1]{\textcolor[rgb]{0.13,0.29,0.53}{\textbf{#1}}}
\newcommand{\DataTypeTok}[1]{\textcolor[rgb]{0.13,0.29,0.53}{#1}}
\newcommand{\DecValTok}[1]{\textcolor[rgb]{0.00,0.00,0.81}{#1}}
\newcommand{\DocumentationTok}[1]{\textcolor[rgb]{0.56,0.35,0.01}{\textbf{\textit{#1}}}}
\newcommand{\ErrorTok}[1]{\textcolor[rgb]{0.64,0.00,0.00}{\textbf{#1}}}
\newcommand{\ExtensionTok}[1]{#1}
\newcommand{\FloatTok}[1]{\textcolor[rgb]{0.00,0.00,0.81}{#1}}
\newcommand{\FunctionTok}[1]{\textcolor[rgb]{0.00,0.00,0.00}{#1}}
\newcommand{\ImportTok}[1]{#1}
\newcommand{\InformationTok}[1]{\textcolor[rgb]{0.56,0.35,0.01}{\textbf{\textit{#1}}}}
\newcommand{\KeywordTok}[1]{\textcolor[rgb]{0.13,0.29,0.53}{\textbf{#1}}}
\newcommand{\NormalTok}[1]{#1}
\newcommand{\OperatorTok}[1]{\textcolor[rgb]{0.81,0.36,0.00}{\textbf{#1}}}
\newcommand{\OtherTok}[1]{\textcolor[rgb]{0.56,0.35,0.01}{#1}}
\newcommand{\PreprocessorTok}[1]{\textcolor[rgb]{0.56,0.35,0.01}{\textit{#1}}}
\newcommand{\RegionMarkerTok}[1]{#1}
\newcommand{\SpecialCharTok}[1]{\textcolor[rgb]{0.00,0.00,0.00}{#1}}
\newcommand{\SpecialStringTok}[1]{\textcolor[rgb]{0.31,0.60,0.02}{#1}}
\newcommand{\StringTok}[1]{\textcolor[rgb]{0.31,0.60,0.02}{#1}}
\newcommand{\VariableTok}[1]{\textcolor[rgb]{0.00,0.00,0.00}{#1}}
\newcommand{\VerbatimStringTok}[1]{\textcolor[rgb]{0.31,0.60,0.02}{#1}}
\newcommand{\WarningTok}[1]{\textcolor[rgb]{0.56,0.35,0.01}{\textbf{\textit{#1}}}}
\usepackage{longtable,booktabs,array}
\usepackage{calc} % for calculating minipage widths
% Correct order of tables after \paragraph or \subparagraph
\usepackage{etoolbox}
\makeatletter
\patchcmd\longtable{\par}{\if@noskipsec\mbox{}\fi\par}{}{}
\makeatother
% Allow footnotes in longtable head/foot
\IfFileExists{footnotehyper.sty}{\usepackage{footnotehyper}}{\usepackage{footnote}}
\makesavenoteenv{longtable}
\usepackage{graphicx}
\makeatletter
\def\maxwidth{\ifdim\Gin@nat@width>\linewidth\linewidth\else\Gin@nat@width\fi}
\def\maxheight{\ifdim\Gin@nat@height>\textheight\textheight\else\Gin@nat@height\fi}
\makeatother
% Scale images if necessary, so that they will not overflow the page
% margins by default, and it is still possible to overwrite the defaults
% using explicit options in \includegraphics[width, height, ...]{}
\setkeys{Gin}{width=\maxwidth,height=\maxheight,keepaspectratio}
% Set default figure placement to htbp
\makeatletter
\def\fps@figure{htbp}
\makeatother
\setlength{\emergencystretch}{3em} % prevent overfull lines
\providecommand{\tightlist}{%
  \setlength{\itemsep}{0pt}\setlength{\parskip}{0pt}}
\setcounter{secnumdepth}{5}
\usepackage{booktabs}
\ifLuaTeX
  \usepackage{selnolig}  % disable illegal ligatures
\fi
\usepackage[]{natbib}
\bibliographystyle{plainnat}
\IfFileExists{bookmark.sty}{\usepackage{bookmark}}{\usepackage{hyperref}}
\IfFileExists{xurl.sty}{\usepackage{xurl}}{} % add URL line breaks if available
\urlstyle{same} % disable monospaced font for URLs
\hypersetup{
  pdftitle={Innovation Handbook},
  pdfauthor={Lucas Nelson},
  hidelinks,
  pdfcreator={LaTeX via pandoc}}

\title{Innovation Handbook}
\author{Lucas Nelson}
\date{December 23, 2023}

\usepackage{amsthm}
\newtheorem{theorem}{Theorem}[chapter]
\newtheorem{lemma}{Lemma}[chapter]
\newtheorem{corollary}{Corollary}[chapter]
\newtheorem{proposition}{Proposition}[chapter]
\newtheorem{conjecture}{Conjecture}[chapter]
\theoremstyle{definition}
\newtheorem{definition}{Definition}[chapter]
\theoremstyle{definition}
\newtheorem{example}{Example}[chapter]
\theoremstyle{definition}
\newtheorem{exercise}{Exercise}[chapter]
\theoremstyle{definition}
\newtheorem{hypothesis}{Hypothesis}[chapter]
\theoremstyle{remark}
\newtheorem*{remark}{Remark}
\newtheorem*{solution}{Solution}
\begin{document}
\maketitle

{
\setcounter{tocdepth}{1}
\tableofcontents
}
\hypertarget{preface}{%
\chapter*{Preface}\label{preface}}
\addcontentsline{toc}{chapter}{Preface}

\begin{center}\rule{0.5\linewidth}{0.5pt}\end{center}

It would be an understatement to say there is \emph{plenty} of buzz surrounding the
Audit Analytics practice. Adopting Databricks has introduced compute resources
that were previously unattainable. The recent class of new hires display a
highly technical skill set and equally passionate ``\emph{innovation-set}''. Our Data
Science neighbors' work using large language models opens up a whole new realm
of revolutionizing audit procedures.

With this buzz comes a lot of great ideas, ideas that - if given adequate
resources - could truly reshape the way our group operates.

The keen reader would have noticed the rather vague phrase ``if given adequate
resources.'' Unfortunately, the field of audit has many self-imposed, yet
arguably reasonable, limitations that negate innovative ideas.

Rather than accept these limitations and self-impose the end of our own ideas,
I'd instead like to advocate for the opposite. Given my time in this group, I
have worked on an array of innovation projects. Some are immediate fixes; others
span many months. Some only reach a few people; others impact the entire
group. Some are interesting; others are very interesting.

\hypertarget{innovation-handbook}{%
\section*{\texorpdfstring{What is the \emph{Innovation Handbook}?}{What is the Innovation Handbook?}}\label{innovation-handbook}}

Innovation projects present their unique challenges and attempt to construct
their unique solutions. The \protect\hyperlink{innovation-handbook}{Innovation Handbook} is an
informal collection of resources dedicated to summarizing, discussing, and
advancing innovation in Audit Analytics. Over the course of the book, I hope to
present a generic template for any innovation project to follow. Some common
themes this book will address include:

\begin{itemize}
\item
  How do I \emph{structure} my innovation project?
\item
  How do innovation projects \emph{connect} to one another?
\item
  How could we define a ``\textbf{successful}'' innovation project?
\end{itemize}

This book could be read cover-to-cover or (more likely) this book could be read
as a reference guide, something to refer to for specific sections from time to
time. Given this book discusses innovation, the content may become outdated
quicker than other books; hence, frequent revisions may be required.

\hypertarget{what-is-the-scope-of-this-book}{%
\section*{What is The Scope of This Book?}\label{what-is-the-scope-of-this-book}}

Let's start with what this book is \textbf{not}.

This book is not a rubric for you to follow. Part of innovating is
discussing and discovering those boundaries that define success on your own.
Although innovation projects could (and arguably should) relate to one another
in some ways, they should actively establish their own boundaries.

This book is not a source of truth. Although I'd like to think that the ideas I
present here have merit, they are the result of my unique experiences. I hope
that while reading this book, you agree and disagree with what I have to say, as
this will lead to productive discussions about how we fundamentally view
innovation in Audit Analytics.

This book is not a manifesto. \emph{No political ideologies will be discussed.}

Onto the good stuff: what this book \textbf{is}.

This book is a philosophical debate. One underlying tone I've observed is that
innovation is put in a faded light, something that some feel comfortable
discussing and few feel comfortable doing. I will endlessly advocate for both
those populations to increase as much as possible.

This book is a long-winded way of saying ``anyone can innovate.'' Just like
Ratatouille's \emph{Anyone Can Cook}, everyone should have the confidence and
resources to pursue their innovative ideas.

This book is a work-in-progress and always will be a work-in-progress. It would
be a shame if a book about innovation was itself not innovative.

\hypertarget{soapbox-my-personal-perspective}{%
\section*{Soapbox: My Personal Perspective}\label{soapbox-my-personal-perspective}}

At the end of some sections and chapters, I have added ``Soapbox'' sections that
give my thoughts in more detail. Typically these ``thoughts'' are synonymous with
``my opinions'' - regardless of the underlying tone of these blocks of text, these
blocks are completely optional (as if the rest of this book isn't).

\hypertarget{about-the-author}{%
\chapter*{About the Author}\label{about-the-author}}
\addcontentsline{toc}{chapter}{About the Author}

\ldots{}

\hypertarget{introduction-to-innovation}{%
\chapter{Introduction to Innovation}\label{introduction-to-innovation}}

\hypertarget{a-brief-history}{%
\section{A Brief History}\label{a-brief-history}}

\emph{What have the last couple years looked like for innovation in Audit Analytics?}

\hypertarget{modern-day}{%
\section{Modern Day}\label{modern-day}}

\emph{Where is that state of innovation in Audit Analytics today?}

\emph{Who/What are the ``sources of innovation''?}

Innovation appeals to everyone in unique ways. Some are interested in the

\hypertarget{before-ai-takes-over}{%
\section{Before AI Takes Over}\label{before-ai-takes-over}}

\emph{How much longer will I have a job before the AI overlords become sentient?}

That title was clickbait.

\hypertarget{the-roles-on-a-team}{%
\chapter{The Roles on a Team}\label{the-roles-on-a-team}}

\hypertarget{a-tangent-stone-carving}{%
\section{A Tangent: Stone Carving}\label{a-tangent-stone-carving}}

\begin{quote}
``Every block of stone has a statue inside it and it is the task of the
sculptor to discover it.''

\hfill --- Michelangelo (supposedly)
\end{quote}

Everyone has something they marvel at every time they comes across it; my marvel
is stone carving.

Stone carving is not an exact art form, but I like to think of it following
three generic steps:

\begin{itemize}
\item
  There is no sculpture without first \textbf{extracting the stone}, carefully
  selecting the stone that best suits the characteristics the sculptor requires.
\item
  There is no sculpture without \textbf{sketching the outline}, determining which
  parts of the rock to chip away and which parts not to chip away given the
  characteristics of the stone.
\item
  There is no sculpture without \textbf{carving the masterpiece}, selecting the
  right tools for the job given the outline sketched previously.
\end{itemize}

Sculptures require these three steps to build off of one another based on
the sculptor's needs. The sculptor should aim to make their most elegant
sculpture given the limitations of the stone they've extracted, the vision they
have in their head, and the tools they have available to them. Once they start
chipping away at the stone, they cannot replace the missing shards, only able
to work their way down to the statue inside the block of stone.

\hypertarget{carving-innovation-stones}{%
\section{\texorpdfstring{Carving \emph{Innovation Stones}}{Carving Innovation Stones}}\label{carving-innovation-stones}}

Unsurprisingly, we do not have stones lying around the office, although it would
be pretty cool if we did. Instead, we have ideas that have yet to be made
clear, ideas that - if given adequate resources - could elevate our existing
processes or introduce new solutions.

\hypertarget{soapbox-discussing-innovation-freely}{%
\section{Soapbox: Discussing Innovation Freely}\label{soapbox-discussing-innovation-freely}}

One outcome of this book is to \textbf{encourage innovation-related discussions}.
Although this section details a rather formal approach for structuring ideas,
most innovation projects (should) start with a simple conversation. Asking
others' what they wish could be different is the perfect path to a two-way
conversation about how you can help make someone's life easier. Who wouldn't
want that?

I do think these conversations lend themselves to a couple pitfalls, some of
which I mentioned above. It is worth reiterating that the primary goals of
these conversations is to conduct ``research'' - you want to gather opinions
from people who can relate to the problem at hand, asking broad questions
that encourage critiques and invite solutions. Avoid coming up with solutions
within the same breath as this can negatively limit the scope of the project
from ``\emph{everything-it-could-possibly-be}'' to
``\emph{anything-you-are-comfortable-with}''.

\hypertarget{parts}{%
\chapter{Parts}\label{parts}}

You can add parts to organize one or more book chapters together. Parts can be inserted at the top of an .Rmd file, before the first-level chapter heading in that same file.

Add a numbered part: \texttt{\#\ (PART)\ Act\ one\ \{-\}} (followed by \texttt{\#\ A\ chapter})

Add an unnumbered part: \texttt{\#\ (PART\textbackslash{}*)\ Act\ one\ \{-\}} (followed by \texttt{\#\ A\ chapter})

Add an appendix as a special kind of un-numbered part: \texttt{\#\ (APPENDIX)\ Other\ stuff\ \{-\}} (followed by \texttt{\#\ A\ chapter}). Chapters in an appendix are prepended with letters instead of numbers.

\hypertarget{footnotes-and-citations}{%
\chapter{Footnotes and citations}\label{footnotes-and-citations}}

\hypertarget{footnotes}{%
\section{Footnotes}\label{footnotes}}

Footnotes are put inside the square brackets after a caret \texttt{\^{}{[}{]}}. Like this one \footnote{This is a footnote.}.

\hypertarget{citations}{%
\section{Citations}\label{citations}}

Reference items in your bibliography file(s) using \texttt{@key}.

For example, we are using the \textbf{bookdown} package \citep{R-bookdown} (check out the last code chunk in index.Rmd to see how this citation key was added) in this sample book, which was built on top of R Markdown and \textbf{knitr} \citep{xie2015} (this citation was added manually in an external file book.bib).
Note that the \texttt{.bib} files need to be listed in the index.Rmd with the YAML \texttt{bibliography} key.

The RStudio Visual Markdown Editor can also make it easier to insert citations: \url{https://rstudio.github.io/visual-markdown-editing/\#/citations}

\hypertarget{blocks}{%
\chapter{Blocks}\label{blocks}}

\hypertarget{equations}{%
\section{Equations}\label{equations}}

Here is an equation.

\begin{equation} 
  f\left(k\right) = \binom{n}{k} p^k\left(1-p\right)^{n-k}
  \label{eq:binom}
\end{equation}

You may refer to using \texttt{\textbackslash{}@ref(eq:binom)}, like see Equation \eqref{eq:binom}.

\hypertarget{theorems-and-proofs}{%
\section{Theorems and proofs}\label{theorems-and-proofs}}

Labeled theorems can be referenced in text using \texttt{\textbackslash{}@ref(thm:tri)}, for example, check out this smart theorem \ref{thm:tri}.

\begin{theorem}
\protect\hypertarget{thm:tri}{}\label{thm:tri}For a right triangle, if \(c\) denotes the \emph{length} of the hypotenuse
and \(a\) and \(b\) denote the lengths of the \textbf{other} two sides, we have
\[a^2 + b^2 = c^2\]
\end{theorem}

Read more here \url{https://bookdown.org/yihui/bookdown/markdown-extensions-by-bookdown.html}.

\hypertarget{callout-blocks}{%
\section{Callout blocks}\label{callout-blocks}}

The R Markdown Cookbook provides more help on how to use custom blocks to design your own callouts: \url{https://bookdown.org/yihui/rmarkdown-cookbook/custom-blocks.html}

\hypertarget{sharing-your-book}{%
\chapter{Sharing your book}\label{sharing-your-book}}

\hypertarget{publishing}{%
\section{Publishing}\label{publishing}}

HTML books can be published online, see: \url{https://bookdown.org/yihui/bookdown/publishing.html}

\hypertarget{pages}{%
\section{404 pages}\label{pages}}

By default, users will be directed to a 404 page if they try to access a webpage that cannot be found. If you'd like to customize your 404 page instead of using the default, you may add either a \texttt{\_404.Rmd} or \texttt{\_404.md} file to your project root and use code and/or Markdown syntax.

\hypertarget{metadata-for-sharing}{%
\section{Metadata for sharing}\label{metadata-for-sharing}}

Bookdown HTML books will provide HTML metadata for social sharing on platforms like Twitter, Facebook, and LinkedIn, using information you provide in the \texttt{index.Rmd} YAML. To setup, set the \texttt{url} for your book and the path to your \texttt{cover-image} file. Your book's \texttt{title} and \texttt{description} are also used.

This \texttt{gitbook} uses the same social sharing data across all chapters in your book- all links shared will look the same.

Specify your book's source repository on GitHub using the \texttt{edit} key under the configuration options in the \texttt{\_output.yml} file, which allows users to suggest an edit by linking to a chapter's source file.

Read more about the features of this output format here:

\url{https://pkgs.rstudio.com/bookdown/reference/gitbook.html}

Or use:

\begin{Shaded}
\begin{Highlighting}[]
\NormalTok{?bookdown}\SpecialCharTok{::}\NormalTok{gitbook}
\end{Highlighting}
\end{Shaded}

\hypertarget{appendix}{%
\chapter*{Appendix}\label{appendix}}
\addcontentsline{toc}{chapter}{Appendix}

The sections below complement other sections throughout the book, elaborating
on previous points or offering my own thoughts that extend the scope of the
book. This material is optional, highly specific, and (it goes without saying)
very opinionated. Read at your own caution.

\hypertarget{the-role-of-the-software-developer}{%
\section*{The Role of the Software Developer}\label{the-role-of-the-software-developer}}

\begin{quote}
\textbf{Soapbox}: This will be the most opinionated section of the book. I have
very strong feelings of how a software developer should contribute to an
innovation project, and even stronger opinions for the opposite.
\end{quote}

\begin{quote}
\textbf{Soapbox}: Nowhere on my LinkedIn page will you find the title ``Software
Developer'', a degree in computer science, or certificates from tech companies.
I only know what I know, and I'm happy to admit there's a lot that I do not
know.
\end{quote}

\hypertarget{innovation-resources}{%
\section*{Innovation Resources}\label{innovation-resources}}

\hypertarget{project-management}{%
\subsection{Project Management}\label{project-management}}

\textbf{Thrills and Chills: Establishing Product Marketing} (\href{https://open.spotify.com/show/4Q4BBL0eJPvRUMaT2rFOUC}{Spotify})

I only started listening to this podcast recently.

\hypertarget{software-development}{%
\subsection{Software Development}\label{software-development}}

Audit Analytics has a \emph{unique} technology stack: Excel, SAS, R, and Databricks.
Listing those on my resume might be seen as a call for help by others.
Subliminal messages aside, the following are resources that software developers
could resort to in the future, broken out by domain-specific categories rather
than language-specific categories (not to be confused with domain-specific-languages.)

Not every topic will entirely cover our group's technology stack. I'll instead
focus on the aspects of our current technology stack that I like and the aspects
of our current technology stack that are unused (that I also like).

\hypertarget{dataframes}{%
\subsubsection{DataFrames}\label{dataframes}}

\hypertarget{python}{%
\subsubsection{Python}\label{python}}

I know I know. ``You said you would stick to domains and not languages.'' You got
me. Let's go ahead and delete this book before anyone else realizes that I am a
liar.

Well, you're still reading so you're okay with the fact that I am a liar. What I
will not lie about is that Python deserves its own section for a couple reasons:

\begin{itemize}
\item
  I could speak highly about it for at least seven minutes uninterrupted.
\item
  It is the language I've worked with the longest and am most comfortable with
  (sorry \texttt{R}, the language used to author this book), and I think most can relate.
\item
  It should be the only language our group utilizes. Period.
\end{itemize}

I do think Python attracts ``\emph{the wrong crowd}'' and is improperly utilized by the
vast majority of people you and I know (myself being one of them). Don't get me
wrong, Python is still a wonderful language that has a supportive community
(despite what I just said). In combination with the technical resources below,
I'll share some non-technical resources that guide my Python philosophy.

\end{document}
